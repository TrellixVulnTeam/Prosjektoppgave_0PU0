\chapter{Cloud Service Providers}
There are hundreds of different cloud service providers. 

\section{Amazon Web Services}
Amazon Web Services is a cloud computing provider that offers a simple way to access servers, storage, databases and a broad set of application services over the Internet \cite{Amazon}. In total, Amazon provides over 50 different solutions where the main services include  Amazon Elastic Compute Cloud (EC2) for compute, which is virtual servers in the Cloud; S3 for storage, providing scalable storage in the cloud; Aurora which is one of several databases, providing a High Performance Managed Relational Database. Amazon also provides several IoT solutions, for instance AWS IoT Platform, which is \textit{a managed cloud platform that lets connected devices easily and securely interact with cloud applications and other devices}.



Amazon Web Services claim to have better cloud security than on-premises infrastructure \cite{Amazon}.

\section{Google}

\section{Microsoft Azure}
Azure is the only consistent hybrid cloud on the market \cite{Azure}. Azure is the cloud for building intelligent applications. Data centers in 42 regions. Recognized as the most trusted cloud for U.S. government institutions, taking advantage of Microsoft security, privacy, and transparency. Ability to run any stack, Linux-based or Windows-based.
\section{IBM}

\section{Firebase}

\section{OpenStack}

\section{Comparison}

\subsection{Azure vs. AWS}
According to Azure \cite{Azure}, they offer more regions than any other cloud provider, they posses unmatched hybrid capabilities and have the strongest intelligence. 
\textit{As the leading public cloud platforms, Azure and AWS each offer businesses a broad and deep set of capabilities with global coverage. Yet many organizations choose to use both platforms together for greater choice and flexibility, as well as to spread their risk and dependencies with a multicloud approach. Consulting companies and software vendors might also build on and use both Azure and AWS, as these platforms represent most of the cloud market demand.} Lists a table to compare AWS with Azure.

\subsubsection{IaaS}
AWS offers a range of tools that fall under IaaS, they are categorized into four classes: content delivery and storage, compute, networking, and database. They all use Amazon's identity and security services while at the same time have a range of management tools that users can use \cite{stackify}. 

Azure on also has four categorize of offerings, namely: data management and databases, compute, networking, and performance. Also Azure provides costumers with security and management tools. 

\subsubsection{PaaS}
AWS does not have as many options or features on the app hosting side. Microsoft has flexed their knowledge of developer tools to have a little bit of an advantage for hosting cloud apps \cite{stackify}.

\subsubsection{Hybrid Cloud}
Hybrid clouds are easier with Azure, partly because Microsoft has foreseen the need for hybrid clouds early on.  Azure offers substantial support for hybrid clouds, where you can use your onsite servers to run your applications on the Azure Stack.  You can even set your compute resources to tap cloud-based resources when necessary. This makes moving to the cloud seamless.  Aside from that, several Azure offerings help you maintain and manage hybrid clouds \cite{stackify}.

\subsubsection{Open Source Developers}
Amazon shines when it comes to open source developers.  Microsoft has historically been very closed to open source applications, and it turned a lot of companies off.  AWS, on the other hand, welcomed Linux users and offered several integrations for open source apps \cite{stackify}.

\subsubsection{Features}
On a per feature basis, you will find that most of all features offered on Azure have a corresponding or similar feature on AWS.  And while it will be quite difficult to come up with an exhaustive features list, you might find it interesting that some Azure services have no AWS equivalent. These include the Azure Visual Studio Online, Azure Site Recovery, Azure Event Hubs, and Azure Scheduler.  However, it seems that AWS is trying to close the gap.  For instance, AWS now offers AWS Lambda on preview to counter Azure’s Logic Apps \cite{stackify}.