\chapter{Implementation of Sensor Network}
 In order to upload sensor data to the cloud, a Raspberry Pi is used as a interface to the cloud. The Raspberry Pi receives data from the sensor nodes via Bluetooth Low Energy (BLE). Since the sensor nodes used in this implementation are not capable of connecting to the Internet, the Raspberry Pi is used as an interface to the cloud. BLE is used as it enables bi-directional communication. More about this specific implementation can be found in the appendix. This chapter discuses a general way to implement a sensor network with BLE as the communication protocol.

\section{Bluetooth Low Energy communication}
BLE has a range of about 100 meters, and periodically send small data packets. It is typically used for applications like monitoring sensors and remote control, among others \cite{BLENordic}. This makes it a very suitable protocol for the application to be implemented in this project. 

Following are some of the key characteristics for BLE, as described by Aftab \cite{AftabBLE}. 

Attribute Protocol (ATT) determines how data is transferred, by defining how devices are discovered as well as reading and writing attributes. Generic Attribute Profile (GATT) is the layer on top of ATT. It defines the role of the device, whether it is a client or a server. A GATT client, usually the central device, requests ATT data from the GATT server, usually one or more peripherals. The GATT server transfers attributes back to the GATT client. Attributes are defined by a Universally Unique Identifier (UUID), a 128-bit ID. A profile is used to group the attributes together, and consists of services and characteristics. A service consists of a collection of multiple characteristics. In this case, the service is Environment Sensing, and the characteristics are sensor measurements like Temperature and Humidity. The characteristics consists of a value and one or more descriptors. I.E. a temperature characteristic could have 24 as the temperature value and Celsius as a descriptor.  

Generic Access Profile (GAP) defines roles as Broadcaster, Observer, Peripheral, and Central used in BLE. A broadcaster sends advertising packets to whoever is listening. Broadcasting packets are generally used to establish a connection between devices. The observers role is to listen to the broadcasting packets. In a master-slave architecture, the central device is the master and the peripheral device is the slave. The central is responsible for making the connections, and can connect to multiple peripherals at once. The peripheral broadcasts advertisement packets for the central to establish connection. The implementation of BLE defines that a device can only have the role as either a peripheral or a central at any given time. 